\section{Compiling application}

The NTU HPC cluster employs the Lmod system to manage the compilers and libraries. This system allows users to easily load a package, switch between different versions of a package, and unload a package. These features are implemented by smartly user's environment variables (e.g. \lstinline{PATH}, and \lstinline{LD_LIBRARY_PATH}) under the hood. The changes are only effective for users' current session.

\subsection{Basic Lmod commands}

To check what are the available package, use

\begin{lstlisting}
$ module av
\end{lstlisting}

\noindent To load a package, use

\begin{lstlisting}
$ module load <packagename>
\end{lstlisting}

\noindent The above command will load the default version for that package (marked with \lstinline{(D)}), to load a specific version, use

\begin{lstlisting}
$ module load <packagename>/<version>
\end{lstlisting}

\noindent To switch the version of a package after it has been loaded, use the above command with the version you want. To check what modules are loaded, use

\begin{lstlisting}
$ module list
\end{lstlisting}

\noindent To unload a module after use, use

\begin{lstlisting}
$ module unload <packagename>
\end{lstlisting}

\noindent To unload all loaded modules at once, use

\begin{lstlisting}
$ module purge
\end{lstlisting}